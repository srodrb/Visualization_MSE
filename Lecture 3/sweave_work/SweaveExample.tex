\documentclass[11pt]{article}
\usepackage{graphicx, verbatim}
\setlength{\textwidth}{6.5in} 
\setlength{\textheight}{9in}
\setlength{\oddsidemargin}{0in} 
\setlength{\evensidemargin}{0in}
\setlength{\topmargin}{-1.5cm}

\usepackage{Sweave}
\begin{document}
\input{SweaveExample-concordance}
\begin{center}
{\bf \Large Sweave Example\\}
 \end{center}

\noindent Simple example
 
\begin{Schunk}
\begin{Sinput}
> a=1
> b=4
> a+b
\end{Sinput}
\begin{Soutput}
[1] 5
\end{Soutput}
\begin{Sinput}
> print("hello")
\end{Sinput}
\begin{Soutput}
[1] "hello"
\end{Soutput}
\end{Schunk}

\noindent We can also include {\tt R code} in the text. Example a+b= 5


%-------------------------------------------
\section{Sweave's options}
%--------------------------------------------

\noindent Show the result, no {\tt R code}:

\begin{Schunk}
\begin{Soutput}
[1] 5
\end{Soutput}
\begin{Soutput}
[1] "hello"
\end{Soutput}
\end{Schunk}

\noindent Do not evaluate the {\tt R} code:

\begin{Schunk}
\begin{Sinput}
> a=1
> b=4
> a+b
> print("hello")
\end{Sinput}
\end{Schunk}


\noindent Evaluate {\tt R} code, do not show results in the console:

\begin{Schunk}
\begin{Sinput}
> a=1
> b=4
> a+b
> print("hello")
\end{Sinput}
\end{Schunk}

NOTE: Let's assume that we are interested in having all chunks with {\tt echo=FALSE, 
results=hide}. This can be indicated in the preamble using {\tt SweaveOpts} :



%-------------------------------------------
\section{Figures}
%--------------------------------------------
A figure can be included in the document by idicating {\tt fig=TRUE}:

\begin{Schunk}
\begin{Sinput}
> plot(1:10, col="red", pch=19)
\end{Sinput}
\end{Schunk}
\includegraphics{SweaveExample-test5}

\newpage

%------------------------------------
\subsection{Anything else about figures}
%-------------------------------------

A nicer figure can be obtained by adding captions, or changing size:

\begin{figure}[h]
\begin{Schunk}
\begin{Sinput}
> par(mfrow=c(1,2))
> plot(1:10, col="green", pch=21)
> barplot(height=sample(1:10,5), names=LETTERS[1:5], col=1:5)
\end{Sinput}
\end{Schunk}
\includegraphics{SweaveExample-test6}
\caption{Figure 1:10 using a barplot inside a 4x6 inches figure}
\end{figure}

\newpage

%------------------------------------
\section{Creating tables}
%-------------------------------------

\noindent Let's include a table using the {\tt women} dataset (included by default in {\tt R})

\begin{Schunk}
\begin{Sinput}
> require(xtable)
> myTable <- summary(women)
\end{Sinput}
\end{Schunk}

\noindent This table can also be included in LaTeX format:


\begin{center}
\begin{tabular}{rrrrrrrr} 

$ Min.   :58.0   $&$ Min.   :115.0   $\\
$ 1st Qu.:61.5   $&$ 1st Qu.:124.5   $\\
$ Median :65.0   $&$ Median :135.0   $\\
$ Mean   :65.0   $&$ Mean   :136.7   $\\
$ 3rd Qu.:68.5   $&$ 3rd Qu.:148.0   $\\
$ Max.   :72.0   $&$ Max.   :164.0   $\\\end{tabular}
\end{center}

\noindent Altough it is even more easy to use the {\tt R} package {\tt xtable}. 

\begin{Schunk}
\begin{Sinput}
> xtab<-xtable(myTable)
> print(xtab, floating=FALSE)
\end{Sinput}
% latex table generated in R 3.0.3 by xtable 1.7-4 package
% Mon Oct  6 08:45:20 2014
\begin{tabular}{rll}
  \hline
 &     height &     weight \\ 
  \hline
1 & Min.   :58.0   & Min.   :115.0   \\ 
  2 & 1st Qu.:61.5   & 1st Qu.:124.5   \\ 
  3 & Median :65.0   & Median :135.0   \\ 
  4 & Mean   :65.0   & Mean   :136.7   \\ 
  5 & 3rd Qu.:68.5   & 3rd Qu.:148.0   \\ 
  6 & Max.   :72.0   & Max.   :164.0   \\ 
   \hline
\end{tabular}\end{Schunk}


%------------------------------------
\subsection{More about tables}
%-------------------------------------

Nicer tables can be created, for instance, exluding the number of rows, or adding a caption. We can also make reference to this
table. Therefore, we reference  Table~\ref{Table:women} in the text:


\begin{Schunk}
\begin{Sinput}
> xtab2<-xtable(myTable, caption="Summary of women data",  
+                        label="Table:women")
> print(xtab2, include.rownames = FALSE)
\end{Sinput}
% latex table generated in R 3.0.3 by xtable 1.7-4 package
% Mon Oct  6 08:45:20 2014
\begin{table}[ht]
\centering
\begin{tabular}{ll}
  \hline
    height &     weight \\ 
  \hline
Min.   :58.0   & Min.   :115.0   \\ 
  1st Qu.:61.5   & 1st Qu.:124.5   \\ 
  Median :65.0   & Median :135.0   \\ 
  Mean   :65.0   & Mean   :136.7   \\ 
  3rd Qu.:68.5   & 3rd Qu.:148.0   \\ 
  Max.   :72.0   & Max.   :164.0   \\ 
   \hline
\end{tabular}
\caption{Summary of women data} 
\label{Table:women}
\end{table}\end{Schunk}


%------------------------------------
\section{Creating pdf}
%------------------------------------

\noindent Just type  (NOTE: this folder must contiain \emph{Sweave.sty} file).

\begin{Schunk}
\begin{Sinput}
> Sweave("SweaveExample.Rnw")
> system("pdflatex SweaveExample.tex")
\end{Sinput}
\end{Schunk}

\noindent Using {\tt Rstudio} is even easier. Just click 'pdf' button. NOTE: pdf must be closed, if not an error message is obtained.


\noindent {\tt cacheSweave} package can be used when computing time is huge. 

\begin{Schunk}
\begin{Sinput}
> library(cacheSweave)
> setCacheDir("cache") # por defecto es "." 
> Sweave("SweaveExample.tex", driver = cacheSweaveDriver)
\end{Sinput}
\end{Schunk}

%------------------------------------
\section{Getting {\tt R} code}
%------------------------------------

\noindent R commands can be obtained in a {\tt .R} file by executing:

\begin{Schunk}
\begin{Sinput}
> Stangle("SweaveExample.Rnw")
\end{Sinput}
\begin{Soutput}
Writing to file SweaveExample.R 
\end{Soutput}
\end{Schunk}

\newpage
%------------------------------------
% Normalmente se incluye esto al final para que sea todo reproducible
%------------------------------------
\section{SessionInfo}
%-------------------------------------

\begin{Schunk}
\begin{Sinput}
>  sessionInfo()
\end{Sinput}
\begin{Soutput}
R version 3.0.3 (2014-03-06)
Platform: x86_64-apple-darwin10.8.0 (64-bit)

locale:
[1] es_ES.UTF-8/es_ES.UTF-8/es_ES.UTF-8/C/es_ES.UTF-8/es_ES.UTF-8

attached base packages:
[1] stats     graphics  grDevices utils     datasets  methods   base     

other attached packages:
[1] xtable_1.7-4

loaded via a namespace (and not attached):
[1] tools_3.0.3
\end{Soutput}
\end{Schunk}
 
\end{document}
